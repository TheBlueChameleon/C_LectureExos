\documentclass[
	ngerman,
	fontsize=10pt,
	parskip=half,
	titlepage=true,
	DIV=12
]{scrartcl}

\usepackage[utf8]{inputenc}
\usepackage{babel}
\usepackage[T1]	{fontenc}
\usepackage{lmodern}
\usepackage{microtype}
\usepackage{color}
\usepackage{csquotes}
\usepackage{amsmath}

\usepackage{minted}
	\usemintedstyle{friendly}

\begin{document}

\part*{C-Kurs, Blatt 02, WiSe 2020}


\section{Taschenrechner}

Programmieren Sie mithilfe der \mintinline{c}{switch} Anweisung einen Mini-Taschenrechner,
der zwei Zahlen und einen Operator (\texttt{+}, \texttt{-}, \texttt{*}, \texttt{/})
einliest und das Ergebnis ausgibt.
Eine eventuelle Division durch 0 soll abgefangen werden.

\emph{Lernziel: switch-case}

\section{Verzinsung}

Schreiben Sie ein Programm, das zu einem gegebenen Anfangskapital
bei einem vorgegebenem jährlichen Zinssatz berechnet,
wie viele Jahre benötigt werden, damit das Kapital eine 
bestimmte Zielsumme überschreitet. 

Mit anderen Worten: Das Programm soll Startkapital, Zielkapital
und den Zinssatz einlesen, und berechen, wie viele Jahre man sparen muss
(\emph{mit Zinseszins!}) bis das Zielkapital erreicht ist.

Lösen Sie diese Aufgabe \emph{ohne} die Verwendung
des Logarithmus (\texttt{log}) und \emph{ohne} die Verwendung
der Potenzfuntion (\texttt{pow}). 

Tipp: Sie benötigen ledlich die Grundrechenarten und \emph{eine} Schleife, am besten \mintinline{c}{while}.

{\em Lernziel: Bedingungen, Schleifen}


\section{Weihnachten}

Geben Sie einen „Weihnachtsbaum“ durch Ausdruck entsprechend vieler
Leerzeichen und Sterne auf dem Bildschirm aus:

\begin{minted}{text}
      *
     ***
    *****
   *******
  *********
 ***********
*************
\end{minted}

Die Aufgabe ist natürlich \emph{nicht} so gedacht, einfach sieben \texttt{printf} zu benutzen, sondern soll mit Schleifen gelöst werden.

Elegant ist es vor allem dann, wenn Sie die Höhe des Weihnachtsbaums variabel halten:

\mintinline{c}{int hoehe = 7;}

Tipp: Es gibt Lösungen mit drei, zwei oder sogar nur einer Schleife. Die einfachste Lösung benutzt drei Schleifen.

\emph{Lernziel: for-Schleifen}

\section{Skalarprodukt}

Berechnen Sie das Skalarprodukt $\vec a \cdot \vec b$ zweier Vektoren,
indem Sie das nachfolgende Programm erweitern. Definition des
Skalarprodukts ist:
%
\[ \vec a \cdot \vec b = \sum_i a_i \cdot b_i \]
%
Sie multiplizeren also jeweils zwei Komponenten mit gleichem Index, und summieren diese Produkte auf.

Beginnen Sie ihren Code folgendermaßen:
\begin{minted}{c}
#include <stdio.h>

int main() {
  double vector_a[7] = { 3, 2, 1, 5, 7, 2, -1};
  double vector_b[7] = {-7, 3, 7, 5, 6, 8,  1};
  /* Bitte ausfuellen */
}
\end{minted}

Tipp: Ergebnis ist $74$ 

\emph{Lernziel: for-Schleife}

\section{Kleingedrucktes}
Lesen Sie mit \mintinline{c}{scanf} einen {\em String} ein, und konvertieren Sie diesen in Kleinbuchstaben. Geben Sie den String danach mit \mintinline{c}{printf} wieder aus.

\emph{Optional}: Zusatzaufgabe: Sorgen Sie dafür, dass auch die Umlaute äöü konvertiert werden.

\emph{Lernziel: Strings, Schleifen}

\section{Telefon-Matrix}
Die Telefon-Matrix ist das folgende Zahlen-Raster:
\begin{align*}
\begin{pmatrix}
	7 & 8 & 9 \\
	4 & 5 & 6 \\
	1 & 2 & 3
\end{pmatrix}
\end{align*}

Legen Sie zwei Arrays an, die diese Daten enthalten. Eines davon soll ein zweidimensionales Array sein (es wird also über \emph{zwei} Indizes angesprochen), das andere ein eindimensionales Array sein (d.\,h. die Tabellen-Elemente werden über eine einzige, fortlaufende Nummer als Index angesprochen.).

Geben Sie die beiden Matrizen auf dem Bildschirm als Tabelle aus.

\emph{Lernziel: Mehrdimensionale Arrays}
\end{document}