\documentclass[
	ngerman,
	fontsize=10pt,
	parskip=half,
	titlepage=true,
	DIV=12
]{scrartcl}

\usepackage[utf8]{inputenc}
\usepackage{babel}
\usepackage[T1]	{fontenc}
\usepackage{lmodern}
\usepackage{microtype}
\usepackage{color}
\usepackage{csquotes}

\usepackage{hyperref}

\usepackage{minted}
	\usemintedstyle{friendly}

\newcommand*{\inC}[1]{\mintinline{c}{#1}}
	
\begin{document}

\part*{C-Kurs, Blatt 01, WiSe 2020}

\section{Arbeitsumgebung Testen}
Falls nicht schon geschehen: Stellen Sie sicher, dass Ihr Computer zum Programmieren in C korrekt eingerichtet ist. Ein Tutorial hierzu ist auf der Kursbeschreibung-Seite der Fakultät Physik verlinkt \url{https://www.ur.de/physik/fakultaet/studium/it-ausbildung/c-und-c/index.html}:
\begin{itemize}
\item Falls Sie Windows benutzen, ist eine ausführliche Beschreibung zu finden unter: \\
	\url{https://www.ur.de/index.php?eID=dumpFile&t=f&f=31172&token=1aa112092627980ac2e917ae7765e65a303b9159} \\
	Zusätzlich sollten Sie den Editor notepad++ installieren.\\
	(Quelle: \url{https://notepad-plus-plus.org/downloads/v7.8.9/})
\item Falls Sie Mac-UserIn sind, installieren Sie das Programm xcode aus dem Appstore.
\item Falls Sie Linux benutzen, ist der gcc bereits installiert und eingerichtet. Installieren ggf. Sie einen Code-Editor wie gedit, kate oder geany. Diese sind in den
	üblichen Repositories frei verfügbar.
\end{itemize}

Laden Sie die Datei \texttt{chameleon.c} herunter, kompilieren Sie diese entweder mit der Zeile
\begin{center}
\texttt{gcc -std=c11 -Wall -Wextra -Wpedantic chameleon.c -o chameleon.exe} (Windows)
\end{center}
oder
\begin{center}
\texttt{gcc -std=c11 -Wall -Wextra -Wpedantic chameleon.c -o chameleon} (Linux und Mac)
\end{center}

Führen Sie das Programm aus.

Falls Sie hierbei Probleme haben, wenden Sie sich an die TutorInnen.
%\clearpage



\section{Hallo!}
Schreiben Sie ein Programm, das den Benutzer begrüßt. Benutzen Sie also \inC{printf}, um eine Nachricht auszugeben.


\section{Operatoren}
Belegen Sie zwei \inC{int}-Variablen mit den Werten 7 und 5 und geben Sie deren Summe, Produkt, Differenz und Quotienten aus. Was müssen Sie tun, damit bei der Division das Ergebnis \texttt{1.400000} angezeigt wird?


\section{Modulo}
Geben Sie mithilfe des Modulo-Operators \texttt{\%} die letzte Ziffer der in einer Variablen gespeicherten Ganzzahl (\inC{int}) aus.

\emph{Hinweis: der Modulo-Operator gibt den Teilungsrest  zurück, also:}
\begin{center}
\texttt{ 9 \% 4 == 1}\\
\texttt{10 \% 4 == 2}\\
\texttt{11 \% 4 == 3}\\
\texttt{12 \% 4 == 0}
\end{center}
(denn 12 ist durch 4 ohne Rest teilbar)

\section{if-else und scanf}
Lesen Sie mit \texttt{scanf} eine \mintinline{c}{float}-Zahl $t$ ein. Prüfen Sie, ob \mbox{$1 + t$} größer als $1$ ist. 

\emph{Optional}: Geben Sie \texttt{1e-20} ein. (Dies ist die Notation für $10^{-20}$.) Warum behauptet Ihr Programm, dass $1 + 10^{-20} \leq 1$?

\emph{Lernziel: if/else}


\section{Schaltjahr}
In unserem Kalender sind zum Ausgleich der astronomischen und kalendarischen
Jahreslänge in regelmäßigen Abständen Schaltjahre eingebaut.
Zur exakten Festlegung der Schaltjahre dienen die folgenden Regeln:

\begin{itemize}
  \item Ist die Jahreszahl durch 4 teilbar,
  so ist das Jahr ein Schaltjahr. Diese Regel hat allerdings eine Ausnahme: 
  \item Ist die Jahreszahl durch 100 teilbar,
  so ist das Jahr kein Schaltjahr. Diese Ausnahme hat 
  wiederum eine Ausnahme: 
  \item Ist die Jahreszahl durch 400 teilbar, so ist das Jahr doch ein Schaltjahr. 
\end{itemize}
Erstellen Sie ein Programm, das berechnet, ob eine vom Benutzer eingegebene Jahreszahl ein 
Schaltjahr bezeichnet oder nicht.

Tipp: Sie können das entweder mit verschachtelten if-else-Anweisungen lösen,
oder mit logischen Operatoren wie \texttt{\&\&} oder \texttt{||}.
Optional können Sie beide Varianten versuchen.

Tipp: Sparen Sie sich Zeit, und verzichten Sie zum Testen zuerst auf die Usereingabe mit \inC{scanf}. Benutzen Sie stattdessen zuerst einen festen Wert \inC{int year = 2020;}, und ändern Sie diese Zeile, um verschiedene Fälle zu testen.

\emph{Lernziel: if/else, Operatoren}


\section{Optional: Darstellung}
Speichern Sie die Zahl $65$ in einer \inC{int}-Variablen,
und geben Sie diese einmal mit dem Format \texttt{\%d},
mit dem Format \verb`%x` und dem Format \verb`%c` aus.
Erhöhen Sie die Variable um $1$ (dann um $2$, umd $3$...)
und geben Sie sie wieder aus. 
Wie könnte man das veränderte Ergebnis erklären?

Tipp: Geben Sie in die Konsole ein: \texttt{man ascii}


\section{Optional: Quadratische Gleichung}
Lösen Sie die quadratische Gleichung $a\,x^2+b\,x+c=0$ für vom Benutzer
(mit \texttt{scanf}) eingegebene Parameter $a$, $b$ und $c$.
Bestimmen Sie, ob die Gleichung überhaupt lösbar ist.
(Wenn Sie mit komplexen Zahlen vertraut sind, können dann
auch die komplexe Lösung ausgeben.)

Die Lösungsformel lautet
\[x_{1,2} = \frac{-b \pm \sqrt{b^2 - 4\,a\,c}}{2\,a}\]

\end{document}
