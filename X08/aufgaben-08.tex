\documentclass[
	ngerman,
	fontsize=10pt,
	parskip=half,
	titlepage=true,
	DIV=12
]{scrartcl}

\usepackage[utf8x]{inputenc}
\usepackage{babel}
\usepackage[T1]	{fontenc}
\usepackage{lmodern}
\usepackage{microtype}
\usepackage{color}
\usepackage{csquotes}
\usepackage{amsmath}
\usepackage{hyperref}

\usepackage{minted}
	\usemintedstyle{xcode}

\usepackage{wrapfig}
\usepackage{graphicx}
\usepackage[bf]{caption} 
	\captionsetup{format=plain}

\usepackage{minted}
	\usemintedstyle{xcode}

\newcommand*{\inC}[1]{\mintinline{c}{#1}}

\begin{document}

\part*{C-Kurs, Blatt 08, WiSe 2020}

Wir bauen in dieser Aufgabe nochmal Schritt für Schritt eine Linked List auf. Unsere Liste soll \inC{int}-Werte verwalten

\section{Datentyp und Listenanfang}
Sehen Sie sich auf Seite 206 des Scripts die \emph{Visualisierung: Verkettete} Liste an, und vergleichen Sie mit dem dort gezeigten Code-Abschnitt:
\begin{minted}{c}
typedef struct listElement_struct {
  int data;
  struct listElement_struct * next;
} listElement_t;
\end{minted}

Machen Sie sich klar:
\begin{itemize}
\item Eine \emph{Instanz} von \texttt{listElement\_t} entspricht in diesem Bild \emph{zwei Zellen}.
\item \emph{Jede Instanz} hat ihre eigenen beiden Attribute (Records). Jedes solche Attribut entspricht in diesem Bild einer Zelle.
\item Wir können in einer Variablen die Daten zu \emph{einer Instanz} speichern. Alle weiteren Instanzen liegen \enquote{irgendwo} im Speicher.
\item Wir können uns aber zu einem beliebigen Element der Liste \enquote{durchhangeln}.
\item Haben wir Beispielsweise die Instanz \texttt{list}, so ist das Nachfolgende Element durch \texttt{*list.next} erreichbar, und der \inC{int}-Wert dieses
	Nachfolgers mit \texttt{(*list.next).data} zugänglich. Zerlegen Sie im Kopf diese Syntax in Ihre Bestandteile, und machen Sie sich klar, welche Zelle(n) im Schaubild
	erreicht werden und warum.
\end{itemize}

Schreiben Sie nun eine Funktion \texttt{newElement(int value)}, die \emph{einen Pointer} auf eine neue Listen-Instanz zurück gibt. Ihre Funktion soll so genutzt werden können:

\begin{minted}{c}
int main () {
  listElement_t * firstElement;
  firstElement = newElement(5);   // legt eine Liste mit einem Element an.
                                  // Der Wert dieses Elements ist 5.
}
\end{minted}

Welchen Rückgabetyp muss also Ihre Funktion haben? Welcher Wert muss für das Attribut \texttt{next} eingetragen werden? Ihre Funktion wird mit dynamischer Speicherverwaltung zuerst Speicherplatz bereit stellen müssen. Denken Sie daran, dass Sie bei der Arbeit mit Pointern und \inC{struct}s die Pfeil-Syntax (\texttt{->}) brauchen werden.

\section{Methode prepend}
Wir wollen nun Elemente an den \emph{Listenanfang} anhängen (also vor das Element mit Wert \inC{5}, das wir schon angelegt haben). Schreiben Sie dazu eine Funktion \inC{void prepend(listElement_t * head, int value)}. Diese soll nach dem folgenden Prinzip funktionieren:

\begin{itemize}
\item Ein neues Element wird mit \texttt{newElement} angelegt.
\item Dieses neue Element soll eine Kopie des ersten Elements sein.
\item Das erste Element der Liste (\texttt{head}) zeigt auf das neu angelegte Element
\item Die \enquote{Nutzdaten} (\texttt{data}) des ersten Elements der Liste (\texttt{head}) erhalten den neuen Wert (\texttt{value})
\end{itemize}

Machen Sie sich auch klar: Auf diese Art ändert sich die Adresse \texttt{firstElement} nicht, die wir in der \texttt{main} verwalten müssen.

\section{Liste Ausgeben}
Wir wollen nun sehen, ob auch wirklich alle Elemente so im Speicher abgelegt wurden, wie Sie dies wollten. Schreiben Sie dazu eine Funktion \inC{showList(listElement_t * head)}. Die Funktion soll die Attribute \texttt{data} jeder Instanz der Linked List auf dem Bildschrim ausgeben. Welchen Rückgabetyp sollte diese Funktion also haben? Woher wissen wir, ob die Liste schon fertig bearbeitet wurde?

\section{Methode pop}
Schreiben Sie eine Funktion \inC{pop(listElement_t ** head)}, die das erste Element einer Liste löscht. Der Rückgabewert soll der Wert des Attributs \texttt{data} des soeben gelöschten Listenelements sein. Denken Sie auch daran, reservierten Speicher mit \texttt{free} freizugeben.

Warum brauchen wir hier einen \texttt{listElement\_t **} statt einem \texttt{listElement\_t *}? (Denken Sie an Ihre Variable \texttt{firstElement} in der \texttt{main}). Welchen Datentyp muss der Rückgabewert haben?

\emph{Hintergrund-Information:} Wenn Sie diese Aufgabe gelöst haben, haben sie eine sogenannte \emph{LIFO-Datenstruktur} implementiert (Last In, First Out; manchmal auch Stack genannt). Wie auf einem Stapel können Sie nun Informations-Einheiten \enquote{übereinander legen} und dann von oben weg wieder abarbeiten. Wo in der Informatik Daten in umgekehrter Reihenfolge des Ansammelns abgearbeitet werden sollen, wird eine solche Stack-Struktur tatsächlich genutzt.

\section{Liste Löschen}
Wie Sie wissen, sollte zu jedem \texttt{malloc} bzw. \texttt{calloc} auch ein \texttt{free} geschrieben werden. Schreiben Sie eine Funktion \inC{clearList(listElement_t * head)}, die den Speicher für \emph{jede Instanz} wieder freigibt. Sie können dazu Ihre Funktion \texttt{pop} benutzen. Erkennen Sie die Ähnlichkeit zur Aufgabe \emph{Liste Ausgeben}?

\section{Optional: Länge der Liste Bestimmen}
Erstellen Sie eine Funktion, die ausgibt, wie viele Elemente die Linked List hat.

\section{Optional: Append}
Schreiben Sie eine Funktion, die ein Element \emph{an das Ende der Liste} anhängt. Erkennen Sie die Ähnlichkeit zur vorigen Aufgabe?

\section{Optional: Element Einfügen}
Schreiben Sie nun eine Funktion, die ein Element \emph{an beliebiger Stelle} in die Liste einfügt. Der Aufruf \texttt{insert(firstElement, 3, 99)} sollte also an Listenplatz 3 ein Element einfügen, und alle anderen Elemente einen Platz nach hinten schieben. (Wie üblich beginnen wir die Zählung bei 0).

Wenn Sie sich geschickt anstellen, können Sie die Funktionen \texttt{append} und \texttt{prepend} benutzen, um diese beiden Spezialfälle zu bearbeiten. Achten Sie auch darauf, dass der User unsinnige Werte eingeben könnte, und fangen Sie diese ab. (Unsinnig wäre z.\;B. der Befehl, ein Element an Stelle \texttt{-1} einzufügen).

\end{document}