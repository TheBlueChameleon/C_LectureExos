\documentclass[
	ngerman,
	fontsize=10pt,
	parskip=half,
	titlepage=true,
	DIV=12
]{scrartcl}

\usepackage[utf8x]{inputenc}
\usepackage{babel}
\usepackage[T1]	{fontenc}
\usepackage{lmodern}
\usepackage{microtype}
\usepackage{color}
\usepackage{csquotes}
\usepackage{amsmath}
\usepackage{hyperref}

\usepackage{minted}
	\usemintedstyle{xcode}

\usepackage{wrapfig}
\usepackage{graphicx}
\usepackage[bf]{caption} 
	\captionsetup{format=plain}

\usepackage{minted}
	\usemintedstyle{xcode}
	
\begin{document}

\part*{C-Kurs, Blatt 09, SoSe 2018}


\subsection*{Worte Zählen}
Schreiben Sie ein Programm, das die Zeichen  und Worte in einer Textdatei zählt. Zum Testen können Sie die Datei \texttt{Praktische\_Physik.txt}\footnote{Quelle: \url{https://www.familie-ahlers.de/wissenschaftliche_witze/aufgaben_zur_praktischen_physik.html}} auf GRIPS benutzen.

Auf Sonderfälle (z.\,B. durch Sonderzeichen im Text) müssen Sie nicht explizit eingehen.

\emph{Optional}:\\
Schreiben Sie das Programm so, dass die zu analysierende Datei als Kommandozeilenparameter übergeben werden kann.\\
Zur Erinnerung: Sie benötigen dann die Signatur:
\mint{c}{int main (int argc, char ** argv)}

\emph{Tipp}: 
Der Header \texttt{string.h} definiert die Funktion \texttt{strlen}.

\emph{Lernziel: Dateien}


\subsection*{Linked Lists}
Sehen Sie sich die Datei \texttt{linkedlist.c} von GRIPS an und vollziehen Sie mit den Vorlesungsfolien nochmals die Funktionsweise der einzelnen Funktionen nach.

Schreiben Sie dann den Funktionskörper zu \mintinline{c}{int ListElementCount (linkedlist_p list);}. Diese Funktion soll die Anzahl der Elemente in der Liste bestimmen. (Der Funktionskopf ist in Zeile 36 bereits angelegt, Sie können hier direkt Ihren Code anfügen.)

Ergänzen Sie ebenso um den Funktionskörper zu \mintinline{c}{int ListRemove (linkedlist_p list, int pos);}. (Auch hier ist der Funktionskopf in Zeile 168 bereits angelegt.)\\
Diese Funktion soll das Listenelement an der Stelle \texttt{pos} entfernen und die Verkettung der übrig gebliebenen Liste entsprechend anpassen. Denken Sie auch daran, mit \mintinline{c}{free} den Speicherbereich für das gelöschte Element wieder freizugeben und achten Sie darauf, dass das erste und letzte Listenelement besonders behandelt werden müssen.\\
Die Funktion soll \texttt{0} ausgeben, wenn das Element erfolgreich aus der Liste entfernt wurde oder einen von \texttt{0} verschiedenen Wert, wenn ein Fehler aufgetreten ist. Fehler könnten z.\,B. sein, dass die Liste weniger als \texttt{pos} Elemente hat.

Testen Sie Ihre Ergänzungen, indem Sie in der \mintinline{c}{int main} Elemente aus Ihrer Liste löschen und an verschiedenen Stellen die Anzahl der Elemente abfragen.

\emph{Optional}\\
Ändern Sie den Aufbau nun so ab, dass Sie die Linked List für einen Routenplaner verwenden können. Dazu müssen Sie einen \mintinline{c}{char *}-Record \texttt{target} und einen \mintinline{c}{double}-record \texttt{arrival}\footnote{Es ist in der IT üblich, Uhrzeiten als \enquote{Prozent des Tages} anzugeben. Um 6 Uhr morgens ist ein Viertel des Tages verstrichen -- dies entspricht also 25\% oder dem double-Wert \texttt{0.25}.} in den Nutzdaten unterbringen. Sie müssen hierzu natürlich einige der bestehenden Funktionen anpassen.

Zur Erinnerung: \texttt{''Text in Double Quotes''} wird zu einem Pointer auf ein \mintinline{c}{char}-Array umgesetzt. Es handelt sich dabei um ein automatisches Array, das nicht mit \texttt{free} wieder freigegeben werden muss.\\
Sie können also die Funktion \mintinline{c}{void myFunc(char * myStringParam);} so aufrufen: \mintinline{c}{myFunc("my String");}

\emph{Lernziel: Linked Lists}


\subsection*{OPTIONAL: Prozessor-Informationen}
Unter Linux finden Sie in der Pseudo-Datei \texttt{/proc/cpuinfo} Informationen über die im System
verbauten Prozessoren.

Bestimmen Sie die Anzahl der Zeilen in dieser Datei, die mit \texttt{processor} beginnen.

\emph{Tipp}: Sehen Sie sich die Funktion \texttt{strncmp} und \texttt{fgets} auf \url{http://de.cppreference.com/w/c} bzw. auf \url{http://en.cppreference.com/w/c} an.

Wenn Sie nicht unter Linux arbeiten, lesen Sie aus der Beispieldatei \texttt{cpuinfo.txt} in GRIPS.

\emph{Lernziel: Dateien}


\subsection*{OPTIONAL: Echter Zufall}
Die Pseudo-Datei \texttt{/dev/urandom} stellt unter Linux und OS~X \emph{echte} Zufallswerte bereit.\footnote{Wie Sie wissen, erzeugt \texttt{rand} lediglich \emph{Pseudozufallszahlen}. Die gleichverteilten Werte entstammen einer mathematischen Sequenz, sind also mit dem ersten Wert beliebig weit vorherbestimmbar.\\
Unixoide Systeme stellen in der genannten Datei einen Datenstrom zur Verfügung, der sich aus der Summe der gesamten Rechneraktivität bildet. Tastatureingaben, Mausbewegung, laufende Prozesse, ... gehen mit ein und sind damit eine gute Quelle für \enquote{echten Zufall}.}

\texttt{/dev/urandom} ist eine Binärdatei. Um eine zufällige \mintinline{c}{int}-Zahl zu erhalten,
lesen sie \emph{binär} aus \texttt{/dev/urandom}, und interpretieren Sie die Daten als \mintinline{c}{int}.

Wenn Sie unter Windows arbeiten, lesen Sie aus der Beispieldatei \texttt{random.bin}, die auf GRIPS zur Verfügung gestellt wurde.

\emph{Lernziel: Dateien}

\end{document}

